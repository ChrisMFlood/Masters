
% !TeX encoding = UTF-8
% !TeX program  = xelatex

\documentclass[10pt,a4paper]{article} 
\usepackage[typeblock=wide]{stb-titlepage}        

\usepackage{calc}

%==== Font setup =====================================================
\usepackage{iftex}
\ifxetex
    \usepackage[math-style=TeX,
                bold-style=TeX,
               ]{unicode-math}
    \setmainfont{Cambria}%........................ Unicode fonts  (Win)                
    \setsansfont[Scale=MatchLowercase]{Calibri}
    \setmonofont[Scale=MatchLowercase]{Consolas}
    \setmathfont{Cambria Math}
    \defaultfontfeatures{Ligatures=TeX}
    \let\bm\symbfit
\else
    \usepackage[utf8]{inputenc}%.................. Unicode file format
    \usepackage{textcomp}%........................ Additional text symbols
    \usepackage[T1]{fontenc}%..................... Type 1 outline fonts
    \usepackage{fourier}%.......................... Roman+math - Utopia
    \usepackage{textcomp}%......................... Additional text symbols
    \usepackage[scaled=0.85]{berasans}%............ Sans serif - Bera sans
    \usepackage[scaled=0.85]{beramono}%............ Typewriter - Bera mono
    \usepackage{bm}%.............................. Bold math fonts
\fi
\normalfont

%==== Tables + Graphics + Color =====================================
\usepackage{array}%............................... Extended table defs 
    \setlength{\extrarowheight}{2pt}
\usepackage{booktabs}
   \heavyrulewidth=\lightrulewidth
   \cmidrulewidth=\lightrulewidth
   \abovetopsep=\smallskipamount
   \belowbottomsep=\smallskipamount
   \cmidrulekern=\tabcolsep
\usepackage{graphicx}%............................ Included graphics
\usepackage[font=small]{caption}%................. Customize captions  
\usepackage[table]{xcolor}%....................... Color setup + colortbl 
    \definecolor{stbMaroon}{RGB}{97,  34, 59}
    \definecolor{stbGold}  {RGB}{183,153, 98}
    \definecolor{stbGreen} {RGB}{130,204,174}
    \definecolor{stbOrange}{RGB}{220, 68,  5}
    \definecolor{stbWine}  {RGB}{166, 10, 61}
    \definecolor{stbSoil}  {RGB}{100, 51, 53}
    
%=====================================================================
\usepackage[pagestyles]{titlesec}%........... Sectioning and headings

\newcommand*\secfamily{\sffamily}
\newcommand*\secfont{\secfamily\bfseries}
\newcommand*\parfont{\normalfont\bfseries\itshape}

\titlelabel{\thetitle.\quad}

\titleformat*{\section}      {\normalfont\Large\secfont}
\titleformat*{\subsection}   {\normalfont\large\secfont}
\titleformat*{\subsubsection}{\normalfont\normalsize\secfont}
\titleformat*{\paragraph}    {\normalfont\normalsize\parfont}
\titleformat*{\subparagraph} {\normalfont\normalsize\parfont\itshape}

\titlespacing*{\section}      {0pt}{3.25ex plus 1ex minus .2ex}{1.5ex plus .2ex}
\titlespacing*{\subsection}   {0pt}{2.5ex  plus 1ex minus .2ex}{1.5ex plus .2ex}
\titlespacing*{\subsubsection}{0pt}{2.5ex  plus 1ex minus .2ex}{1.5ex plus .2ex}
\titlespacing*{\paragraph}    {0pt}{1.5ex  plus 1ex minus .2ex}{1em}
\titlespacing*{\subparagraph} {\parindent}{1.5ex  plus 1ex minus .2ex}{1em}


%=====================================================================
\usepackage[]{url}


\usepackage{dirtree}
    \newlength\Loffset \setlength\Loffset{3.0em}
    \newlength\Lwidth  \setlength\Lwidth{1em}
    \newlength\Lsep    \setlength\Lsep{0.2em}
    \newlength\Lskip   \setlength\Lskip{\Loffset+\Lwidth}
    \DTsetlength{\Loffset}{\Lwidth}{\Lsep}{0.4pt}{1.6pt}%
    \newsavebox{\DirBox}

\usepackage[shortlabels]{enumitem}


%==== User Defs ======================================================
\makeatletter

\newsavebox{\tboxa}
\newlength{\tdima}

% Ensure that the minus sign is the "-" character in
% listings environments for cut and paste operations from pdf docs
\newcommand*\verbfont{%
    \normalfont\ttfamily\hyphenchar\font\m@ne\@noligs}
\def\UrlFont{\verbfont}

\newenvironment{fminipage}[1][\linewidth]
    {\setlength{\tdima}{#1-2\fboxsep-2\fboxrule}%
     \begin{lrbox}{\tboxa}\begin{minipage}{\tdima}}%
    {\end{minipage}\end{lrbox}\noindent\fbox{\usebox{\tboxa}}}

\newenvironment{cminipage}[1][\linewidth]
    {\setlength{\tdima}{#1-2\fboxsep-2\fboxrule}%
     \begin{lrbox}{\tboxa}\begin{minipage}{\tdima}}%
    {\end{minipage}\end{lrbox}\noindent\colorbox{stbMaroon!10!white}{\usebox{\tboxa}}}

\newenvironment{Cpara}[1][\normalsize]
    {\list{}{\setlength{\leftmargin}{0pt}%
             \setlength{\labelwidth}{0pt}%
             \setlength{\labelsep}{0pt}}%
     \item\relax\cminipage\small#1\ignorespaces}%
    {\endcminipage\endlist}

\newlength\mylena
\newlength\mylenb
\newcommand\mystrut[1][4pt]{%
    \setlength\mylena{\totalheightof{\@arstrutbox}+#1*2}%
    \setlength\mylenb{\depthof{\@arstrutbox}+#1}%
    \rule[-\mylenb]{0pt}{\mylena}}

\newlength{\mytab}
\setlength{\mytab}{2\parindent}
\newcommand{\tab}{\hspace*{\mytab}}

\newcommand\ctab{\makebox[\mytab][l]{\quad\texttt{\textgreater}}}

\newenvironment{CMDline}[1][\normalsize]
    {\list{}{\setlength{\leftmargin}{0pt}%
             \setlength{\labelwidth}{0pt}%
             \setlength{\labelsep}{0pt}}%
     \item\relax%
          \begin{tabular}{@{}p{\linewidth}@{}}
               \hline\rowcolor{stbMaroon}[0pt][0pt]
               \mystrut\color{white}\small\ctab\ttfamily#1\ignorespaces}%
    {\\\hline\end{tabular}\endlist}

\newcommand*\lcbrack{\symbol{`\{}}
\newcommand*\rcbrack{\symbol{`\}}}
\newcommand*\bslash{\symbol{`\\}}
\newcommand*\uscore{\symbol{`\_}}

\newcommand*\file[1]{\,\texttt{#1}\,}%{\texttt{ʺ#1ʺ}}
\newcommand*\prgm[1]{\textsf{#1}}

\newcommand\TL{\prgm{\TeX{}Live}}
\newcommand\MT{\prgm{Mik\TeX}}

\newcommand\pkg[1]{\textsf{#1}}

\makeatother

%==== Title Page =====================================================
\title{Stellenbosch-2 bundle}                   
\author{Danie Els}                
\address{Department of Mechanical and Mechatronic Engineering\\
        Stellenbosch University \\
        Private Bag X1, Matieland 7602, South Africa}
\date{2023/08/10}                             
%\Copyright{2023}{Stellenbosch University.\\ All rights reserved.}

\begin{document}
\maketitle 

\noindent\begin{fminipage}
\begin{Cpara}
The Stellenbosch-2 bundle is provided to typeset reports, theses and
dissertations and presentations that conform to the requirements of the Stellenbosch
University.\medskip

It is distributed with standard \TL\ and \MT\
systems and there is no need to install it separately if you have a full
installation. All the package updates and error fixes can then be obtained
with the normal system updates.\medskip

\emph{The original Stellenbosch/usthesis packages will remain on CTAN for typesetting of legacy documents.}
\end{Cpara}
\end{fminipage}\smallskip


\section*{Packages and files}

\begin{description}[style=nextline]
    \item[\pkg{stb-thesis}:] A class that is primarily concerned with the
        formatting of Stellenbosch University thesis and dissertation front matter such as the title page, abstract, etc. It also works together with the babel package
        to provide language language options to typeset documents in
        Afrikaans or in English. The class file \file{stb-thesis.cls} is based on the standard \LaTeX\ book class.

    \item[\pkg{stb-bib}:]  A  \prgm{Bib\TeX} package for the formatting of
        bibliographic references of theses. This package is tailored
        towards citations and bibliographical formatting for the natural
        sciences and engineering.

    \item[\pkg{stb-titlepage}:]  A package that redefine the standard title page of the
        \LaTeX\ classes to add a logo at the top and an address line below
        the author. It is intended for use when writing general notes and
        articles.

    \item[\pkg{stb-beamer}:]  Style files to produce presentations with the Stellenbosch University logos and colours.
   
    \item[\pkg{stb-nomencl}:]  Simple utility to set a nomenclature or list of
        symbols.

    \item[\texttt{templates}:] Templates are provide for Stellenbosch theses, dissertations and presentations as well as for a final year M\&M project. 
\end{description}


\noindent You can test if the Stellenbosch bundle is installed by searching
for one of the style files with the \TeX\ command line utility:
\begin{CMDline}
     kpsewhich stb-thesis.cls
\end{CMDline}
If is returns the path to \file{stb-thesis.cls} then the bundle is installed
and \TeX\ can find it.


\section*{Documentation}

The documentation of the Stellenbosch thesis bundle can be found at
\begin{Cpara}
    \tab\url{https://www.ctan.org/pkg/stellenbosch-2}
\end{Cpara}
\medskip

\noindent If you have full \LaTeX\ installation then the documentation can be
obtained with the command line utility
\begin{CMDline}
    texdoc -l stellenbosch-2
\end{CMDline}


\section*{Local installation}

If for one or other reason you do not have a full \LaTeX\ installation and
needs to install the Stellenbosch thesis bundle on you machine, you can
download the full package from CTAN:
\begin{Cpara}
    \tab\url{https://ctan.org/pkg/stellenbosch-2}
\end{Cpara}
Note that all the files needed to be installed in the correct directory
structure so that the \TeX\ search engines can find them. The files can be
installed in the \texttt{TEXMFHOME} directory for the current user or
\texttt{TEXMFLOCAL} directory for all the users. These directories are
created when your \TeX\ system is installed and can be found by running the
command line utility:
\begin{CMDline}
    kpsewhich -var-value TEXMFHOME
\end{CMDline}
\begin{CMDline}
    kpsewhich -var-value TEXMFLOCAL
\end{CMDline}

\noindent Unzip \file{stellenbosch-2.tds.zip} to the selected directory and
make sure that the unzipped directory structure stays intact. The files must
be in the structure as indicated below with \texttt{$\langle$TEXMF$\rangle$}, the \texttt{TEXMFHOME} or \texttt{TEXMFLOCAL} directory:

\begin{Cpara}
        \begin{lrbox}{\DirBox}%
        \begin{minipage}{\linewidth}%
        \dirtree{%
        .1 \quad$\langle$TEXMF$\rangle$.
        .2 \bslash bibtex\bslash bst\bslash stellenbosch-2.
        .4 stb-bib-eng-a.bst.
        .4 stb-bib-eng-n.bst.
        .2 \bslash doc\bslash latex\bslash stellenbosch-2.
        .4 stellenbosch-2.pdf.
        .4 stb-beamer.pdf.
        .4 stb-bib.pdf.
        .4 stb-nomencl.pdf.
        .4 stb-thesis.pdf.
        .4 stb-titlepage.pdf.
        .4 \bslash templates.
        .5 template-stb-article.zip.
        .5 template-stb-beamer.zip.
        .5 template-stb-masters.zip.
        .5 template-stb-report-eng.zip.
        .2 \bslash tex\bslash latex\bslash stellenbosch-2.
        .4 stb-beamer-a.sty.
        .4 stb-beamer-b.sty.
        .4 stb-thesis.cls.
        .4 stb-thesis.afr.
        .4 stb-thesis.eng.
        .4 stb-bib.sty.
        .4 stb-bib.afr.
        .4 stb-bib.eng.
        .4 stb-titlepage.sty.
        .4 stb-nomencl.sty.
        .4 \bslash figs.
        .5 All the graphics....
        }
        \end{minipage}%
        \end{lrbox}%
        \tab\hspace*{-\Lskip}\usebox{\DirBox}
\end{Cpara}
\bigskip

\noindent The final step is to update the \TeX\ file name database so that
all the files can be found by the search engines. Run the following command
line utility (it may take a while):

\begin{CMDline}
    mktexlsr
\end{CMDline}
or
\begin{CMDline}
    texhash
\end{CMDline}


\noindent Finally to see if your the bundle is installed correctly, compile
one of the provided templates.





\end{document}