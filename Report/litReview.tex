\chapter{Lit Review}



Autonomous driving requires accurate information about the vehicle pose and motion state in order to achieve precise tracking of the planned trajectory. In this paper we propose a robust architecture to localize a high performance race car and show experimental results with speeds up to 150 $\frac{km}{h}$ and utilizing approximately 
80 $\%$ of the available friction level. The concept has been applied using the development vehicle DevBot taking part in the Roborace competition. To achieve robust and reliable performance, we use two independent localization pipelines, one based on GPS and one on LIDARs. We propose to fuse them via a Kalman Filter based on a purely kinematic model and show, that it can outperform a high fidelity model under realistic race conditions. An outstanding property of this concept is that it does not depend on any of the vehicles parameter and is therefore robust to varying tire and track conditions. Further we present an adaption method for the measurement covariances based on the track layout. This allows to combine the strengths of each localization method. \cite{Alex19}